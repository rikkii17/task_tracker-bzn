\documentclass[11pt, a4paper, jf]{jarticle} % jarticleを使用し、日本語組版環境に依存

% --- 日本語環境安定化のための最小限のパッケージ ---
% otf, uprefなど環境依存性の高いパッケージを削除

\usepackage{amsmath} % 数式用
\usepackage{amssymb} % \squareや\checkmarkを使うために必要

% ジオメトリ設定はそのまま
\usepackage[a4paper, top=2.5cm, bottom=2.5cm, left=2cm, right=2cm]{geometry} 

\usepackage{enumitem}

% リストの行頭にチェックボックス記号を配置 (数式モードを維持)
% jarticle環境ではjfオプションによりフォントは自動で処理されます
\setlist[itemize]{label={$\square$}, leftmargin=*}
% --- END 最小限のパッケージ ---

\title{\textbf{Task Tracker 7日間開発計画:詳細チェックリスト}}
\author{(開発者名またはプロジェクトチーム名)}
\date{2025年11月}

\begin{document}
\maketitle
\thispagestyle{empty} % 1ページ目のページ番号を非表示

\section*{はじめに}
本チェックリストは、提供された「Task Tracker 開発要求仕様書」に基づき、React/NestJS/Prisma/JWT を使用したアプリケーションを7日間(1週間)で完成させるための実行計画である。

\vspace{0.5cm}

\textbf{目標技術スタック:}
\begin{itemize}
    \item フロントエンド: React, TypeScript, Tailwind CSS
    \item バックエンド: NestJS, TypeScript, JWT (認証)
    \item データベース: PostgreSQL (Supabase), Prisma (ORM)
\end{itemize}


\section{一日目:環境構築とDB設計(バックエンドの準備)}

\textbf{目標:モノレポのセットアップとデータ構造の確立}

\begin{itemize}
    \item モノレポ構造の作成(\texttt{root/}、\texttt{client/}、\texttt{server/} ディレクトリ)。
    \item Gitリポジトリの初期化と基本ファイルのコミット。
    \item NestJS (\texttt{server/}) プロジェクトの初期設定(Prisma含む)。
    \item SupabaseでPostgreSQLデータベースを作成し、接続情報を取得。
    \item Prismaスキーマ(\texttt{User}、\texttt{Task} モデル、\texttt{TaskStatus} Enum)を仕様書通りに定義。
    \item Prismaマイグレーションを実行し、DBにテーブルを反映(\texttt{npx prisma migrate dev})。
\end{itemize}

\section{二日目:バックエンド認証機能の実装}

\textbf{目標:ユーザー登録、ログイン、JWT認証フローの確立}

\begin{itemize}
    \item \texttt{AuthModule} と \texttt{UsersModule} を作成。
    \item JWT Strategy、Passport Guard のセットアップ。
    \item サインアップ (\texttt{POST /auth/signup}) エンドポイントの実装。
    \begin{itemize}
        \item パスワードのハッシュ化(\texttt{bcrypt}など)を必須とする。
    \end{itemize}
    \item ログイン (\texttt{POST /auth/login}) エンドポイントの実装。
    \begin{itemize}
        \item パスワード検証とJWT(アクセストークン)の生成。
        \item レスポンスボディでトークンをクライアントに返す。
    \end{itemize}
    \item \texttt{AuthGuard} を適用したエンドポイントを仮作成し、認証が機能することを確認。
\end{itemize}

\section{三日目:バックエンド タスクCRUD(一覧・作成)}

\textbf{目標:タスク管理の核となるエンドポイントの実装}

\begin{itemize}
    \item \texttt{TasksModule} を作成し、\texttt{AuthGuard} を適用。
    \item タスク新規登録 (\texttt{POST /tasks}) エンドポイントの実装。
    \begin{itemize}
        \item \texttt{title} が必須であることを検証。
        \item 認証情報から \texttt{userId} を自動で紐づけるロジックを実装。
        \item 初期 \texttt{status} を \texttt{TODO} に設定。
    \end{itemize}
    \item タスク一覧取得 (\texttt{GET /tasks}) エンドポイントの実装。
    \begin{itemize}
        \item ログイン中のユーザーが所有するタスクのみをフィルタリング。
        \item 更新日時順などでソートして返す(推奨)。
    \end{itemize}
\end{itemize}

\section{四日目:バックエンド タスクCRUD(編集・削除)と認可}

\textbf{目標:タスクの編集・削除と所有権チェック(認可)の完了}

\begin{itemize}
    \item タスク編集 (\texttt{PATCH /tasks/:id}) エンドポイントの実装。
    \item タスク削除 (\texttt{DELETE /tasks/:id}) エンドポイントの実装。
    \item \textbf{所有権チェック(認可)} ロジックの徹底実装。
    \begin{itemize}
        \item 取得、編集、削除の全てのエンドポイントで、リクエストユーザーIDとタスクの\texttt{userId}の一致を厳密に検証。
    \end{itemize}
    \item バックエンド全体のロジックとAPI仕様(第4章)の最終チェック。
\end{itemize}

\section{五日目:フロントエンド環境構築と認証UI/UX}

\textbf{目標:React環境のセットアップと認証画面の完成}

\begin{itemize}
    \item React (\texttt{client/}) プロジェクトの初期設定 (TypeScript)。
    \item Tailwind CSS の設定と基本レイアウトの作成。
    \item 共有型定義 (\texttt{Task}、\texttt{TaskStatus}) を \texttt{client/src/types/} に定義。
    \item 認証画面(サインアップ、ログイン)のUI/UX実装。
    \item \texttt{Axios} を使用した BE認証APIへの接続ロジックを実装。
    \item ログイン成功後、JWT(アクセストークン)を保持し、認証状態を管理するContext/Hookを作成。
\end{itemize}

\section{六日目:フロントエンド タスクCRUD UIの実装}

\textbf{目標:タスク管理のメインUIと操作ロジックの完成}

\begin{itemize}
    \item 認証後のメイン画面(タスク一覧)のレイアウトとコンポーネントを作成。
    \item \texttt{GET /tasks} をコールし、取得したタスク一覧を表示するロジックを実装。
    \item タスク新規登録モーダル/フォームの実装(\texttt{POST /tasks})。
    \item タスクの編集機能の実装(タイトル、詳細、ステータスの更新、\texttt{PATCH /tasks/:id})。
    \item タスク削除機能の実装と確認モーダル。
    \item \textbf{モバイルフレンドリーなレスポンシブ対応}を完了。
\end{itemize}

\section{七日目:統合、デプロイ、提出準備}

\textbf{目標:最終動作確認、デプロイ、提出要件の完了}

\begin{itemize}
    \item フロントエンドとバックエンドの統合テスト(サインアップからCRUDまで)。
    \item \textbf{Render} への NestJS API のデプロイと環境変数の設定。
    \item \textbf{Vercel} への React アプリケーションのデプロイと環境変数の設定。
    \item エラーハンドリング(認証失敗、タスク未検出など)の最終チェックと改善。
    \item \texttt{README.md} の作成/更新(セットアップ手順、使用技術一覧、デプロイURL、API仕様書)。
    \item GitHub Public リポジトリの確認と最終コミット。
    \item 評価観点(設計力、実装力、コード品質、UI)に沿ったセルフレビュー。
\end{itemize}

\end{document}